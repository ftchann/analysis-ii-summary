\subsection{Wegintegrale}

Sei $f: [a, b] \rightarrow \R^{n}$ stetig, d.h. für
\[ f(t) = (f_1(t), ~ \ldots, ~ f_n(t)) \]
jedes $f_i$ stetig, dann ist
\[ \int_a^b f(t) dt = \left( \int_a^b f_1(t) dt, ~ \ldots , ~ \int_a^b f_n(t) dt\right) \in \R^n \]

Für eine parametrisierte Kurve in $\R^n$, d.h.
$\gamma : [a, b] \rightarrow \R^n$, s.d.
\begin{enumerate}
\item{ $\gamma$ stückweise stetig}
\item{2. $\exists t_0, \ldots, t_k$, ~ s.d. $t_0 = a < t_i < t_k = b$, ~~ s.d. ~
$\gamma ~ | ~ ]t_i, t_{i-1}[ ~ \in C^1$}
\end{enumerate}
nennen wir $\gamma$ einen Pfad zwischen
$\gamma(a)$ und $\gamma(b)$. 

\begin{Satz}{Länge einer Kurve}{}
	Sei $\gamma$ eine reguläre Kurve $t \to \gamma(t)$ sei $|.|$ die euklidische Norm: Die Länge ist
	\[
  		L(\gamma) = \int_a^b |\gamma'| dt
  	\]
\end{Satz}

\begin{Rezept}{Wegintegrale}{}
	Gegeben: Vektorfeld f von der Klasse $C^1$ und eine Kurve $\gamma \in C^1_{pw}$.
	
	Gesucht: Wegintegral $\int_\gamma f \cdot ds$.\\
	
	\textbf{Lösungsschritt I:}
	
	Parametrisiere $\gamma$, d.h. finde eine Abbildung $\gamma(t) : [a, b] \to \R^n$, $t \to \gamma(t)$.
	
	\textbf{Lösungsschritt II:}
	
	Berechne $\gamma'(t) = \frac{d}{dt} \gamma(t)$. Dabei wird jede Komponente des Vektors $\gamma$ einzeln nach t abgeleitet.
	
	\textbf{Lösungsschritt III:}
	
	Das Wegintegral von $f$ entlang $\gamma$ ist definiert als
	\[
		\int_{\gamma} f(s)\cdot d\vec{s} = 
		\int_a^b \underbrace{\underbrace{f(\gamma(t))}_{\in\R^n}  \cdot 
		\underbrace{\gamma'(t)}_{\in\R^n}}_{\text{Skalarprodukt in } \R} dt  ~
		\in \R
	\]
	und ist unabhängig der gewählten Parametrisierung!
\end{Rezept}

\begin{Definition}{Vektorfeld}{}
	Für $X\subset \mathbb{R}^n$, $f:X\rightarrow \mathbb{R}^n$ wird \textbf{Vektorfeld} genannt. 
\end{Definition}

\begin{Satz}{Orientierte Reparametriesierung}{}
	Sei $\gamma:[a,b] \rightarrow \mathbb{R}^n$ eine parametrisierte Kurve und $\sigma:[c,d] \rightarrow \mathbb{R}^n$ eine orientierte Reparametriesierung
	so dass $\sigma =\gamma \circ \varphi$, wobei 
	$\varphi : [c,d] \rightarrow [a,b]$ stetig differenzierbar auf $
	]a,b[$ ist und zudem streng monoton steigend und $\varphi(a) = c$ sowie $\varphi(b)=d$.\\

	Sei $X$ das Bild von $\gamma$, oder ist das equivalente Bild von $\sigma$ und $f$ stetig.
	Dann gilt \textcolor{red}{$\int_{\gamma}f(s)\cdot d\vec{s} = \int_{\sigma}f(s)\cdot d\vec{s}$}
\end{Satz}
