\subsection{Das Riemann Integral}

\begin{Rechenregeln}{Riemann Integral}{}
    Das Riemann Integral über ein Quader $Q = [a_1,a_2] \times [a_2, b_2] \times \dots \times [a_n, b_n]$ wird mittel Partitionen Unter und Obersummen, wie im 1-Dimensionalen Fall definiert.
    Das Untere und Obere Riemann Integral ist:
    \[ 
        \int_{Q} f(x) d \mu=\underline{I}(f):=\sup \left\{U_{f}(P) \mid P \in \mathcal{P}(Q)\right\}
    \]
        
    \[
        \left(\text{resp ., } \int_{Q} f(x) d \mu=\bar{I}(f):=\inf \left\{O_{f}(P) \mid P \in \mathcal{P}(Q)\right\}\right)    
    \]
    $f$ heisst integrierbar wenn, $\bar{I}(f) = \underline{I}(f)$
\end{Rechenregeln}

\begin{Satz}{Eigenschaften des Riemann Integrals}{}
    \begin{enumerate}
        \def\labelenumi{\arabic{enumi}.}
        \item
          \textbf{Compatibility:} If \textbf{\(n=1\)} and \(X=[a,b]\) with
          \(a\leq b\), then \(\int_{[a,b]}f(x)dx=\int_a^bf(x)dx\).
        \item
          \textbf{Linearity:} If \(f\) and \(g\) are continuous on \(X\) and
          \(a,b \in \mathbb{R}\), then
          \(\int_X(af_1(x)+bf_2(x))dx = a\int_Xf_1(x)dx+b\int_Xf_2(x)dx\).
        \item
          \textbf{Positivity:} If \(f\leq g\), then
          \(\int_Xf(x)dx\leq\int_Xg(x)dx\) and especially, if \(f \geq 0\), then
          \(\int_Xf(x)dx \geq 0\). Moreover, if \(Y \subset X\) is compact and
          \(f\geq 0\), then \(\int_Yf(x)dx\leq\int_Xf(x)dx\).
        \item
          \textbf{Upper bound and triangle inequality:} In particular, since
          \(-|f|\leq f \leq |f|\), we have
          \(\left| \int_Xf(x)dx \right| \leq \int_X|f(x)|dx\), and since
          \(|f+g| \leq |f| + |g|\), we have
          \(\left| \int_X(f(x)+g(x))dx \right| \leq \int_X|f(x)|dx + \int_X|g(x)|dx\).
        \item
          \textbf{Volume:} If \(f=1\), then the integral of \(f\) is the
          \emph{volume} in \(\mathbb{R^n}\) of the set \(X\), and if \(f\geq 0\)
          in general, the integral of \(f\) is the volume of the set
          \(\{(x,y) \in X \times \mathbb{R} : 0 \leq y \leq f(x)\} \subset \mathbb{R}^{n+1}\).
          In particular, if \(X\) is a bounded \emph{rectangle}, say
          \(X = [a_1,b_1] \times ... \ \times [a_n, b_n] \subset \mathbb{R}^n\)
          and if \(f=1\), then \(\int_Xdx=(b_n-a_n) \cdot\cdot\cdot(b_1-a_1)\).
          We write $\text{Vol}(X)$
        \end{enumerate}
  
\end{Satz}
\begin{Satz}{Satz von Fubini}{}
    Reduzierung von mehrdimensionalen Integralen auf eine Dimension. Sei $f: [a,b] \times [c, d]$ stetig, dann gilt
    \[ \int_a^b dx \int_c^d f(x, y) dy = \int_c^d dy \int_a^b f(x, y) dx = \int\displaylimits_{[a,b] \times [c, d]} f(x, y) d\mu(x, y)   \]

    Es sei der Quader $Q = [a_1,a_2] \times [a_2, b_2] \times \dots \times [a_n, b_n]$ mit $f \in C^0(Q)$ gegeben. Dann gilt

    \[
        \int_Q f(x) d\mu(x) = \int_{a_1}^{b_1} dx_1 \int_{a_2}^{b_2} dx_2 \dots \int_{a_n}^{b_n} dx_n f(x_1, x_2,...,x_n)
    \]
    
    \textbf{Die Integrationsreihenfolge darf vertauscht werden.}
\end{Satz}

\begin{Definition}{Vernachlässigbare Mengen}{}
    \begin{enumerate}
        \def\labelenumi{\arabic{enumi}.}
        \item
          Let \(1\leq m \leq n\) be an integer. A parameterised \(m\)-set in
          \(\mathbb{R}^n\) is a continuous map
          \(f:[a_1,b_1]\times ...\ \times[a_m,b_m] \rightarrow \mathbb{R}^n\)
          which is \(C^1\) on \(]a_1, b_1[\ \times ... \times\ ]a_m,b_m[\).
        \item
          A subset \(B\subset \mathbb{R}^n\) is negligible if there exists an
          integer \(k \geq 0\) and parameterised \(m_i\)-sets
          \(f_i:X_1 \rightarrow \mathbb{R}^n\), with \(1\leq i \leq k\) and
          \(m_i < n\), such that \(X\subset f_1(X_1) \cup ...\ \cup f_k(X_k)\) .
        \item Satz: Wenn $X$ vernachlässigbar ist gilt:
            $\int_X f(x) dx = 0$
    \end{enumerate}
\end{Definition}
\begin{Definition}{Uneingentliche Integrale}{}
    Sei $X$ nicht kompakt. Und $X_k$ eine Reihe von kompakten Mengen sodass $X_k \subset X_{k+1}$ und
    $\cup_{k = 1}^{\infty} = X$. Dann konvergiert $\int_X f dx$ wenn
    \[
        \int_X f dx = \Lim{k \to \infty}{\int_X f dx}
    \]
    Im 2-Dimensionalen Fall, wenn die Anwendung von Fubini gleich ist.
\end{Definition}

\begin{Satz}{Substitutionsregeln in einer Dimension}{}
    Sei $f$ eine Riemann-integrierbare Funktion. Für die Berechnung des Integrals
    \[
        \int_a^b f(x) dx
    \]
    führt die Substitution $x \to g(u)$ zu $dx = g'(u)du$ und damit wird das Integral
    \[
        \int_a^b f(x) dx = \int_{g^{-1}(a)}^{g^{-1}(b)} f(g(u)) g'(u) du
    \]
    Das heisst wir haben das Integrationselement $dx$ durch $g'(u)du$ ersetzt und die Grenzen entsprechend angepasst.
\end{Satz}

\begin{Satz}{Substitutionsregel in $\R^2$}{}
    \[
        \int_{\Omega} f(x,y) ~ dx dy = \int_{\widetilde{\Omega}} f(g(u, v), \; h(u, v)) \, 
		    \left\lvert\det \, \underbrace{\begin{pmatrix}
			    \frac{\partial g}{\partial u} & \frac{\partial g}{\partial v}\\
			    \frac{\partial h}{\partial u} & \frac{\partial h}{\partial v}
		    \end{pmatrix}}_{d\Phi = \nabla\Phi} \right\rvert
	    ~ du dv
    \]
\end{Satz}

\begin{Satz}{Substitutionsregeln in $n$ Dimensionen}{}
    Sei $f$ eine Riemann-integrierbare Funktion auf dem Gebiet $\Omega \subset \R^n$ und die Koordinatentransformation (Substitution)
    \[
    (x_1,\hdots,x_n) = \Phi(u_1, \hdots,  u_n)
    \]
    oder in Komponenten
    \[
        \vek{x_1}{\vdots}{x_n}
        = \Phi(u)
        = \vek{g_1(u_1,\hdots,u_n)}{\vdots}{g_n(u_1,\hdots,u_n)}
    \]
    ist ein $C^1$-Diffeomorphismus. Dann gilt
    \[
        \int_\Omega f(x_1, \hdots, x_n)dx_1\hdots dx_1 = \int_{\widetilde{\Omega}} f(g_1(u), \hdots, g_n(u)) \cdot |\det d \Phi|\ du_1\hdots du_n
    \]
    wobei das Gebiet $\widetilde{\Omega} = \Phi^{-1}(\Omega)$ ist. $|\det d\Phi|$ ist die \textbf{Funktionaldeterminante} (Jakobi-Determinante).
\end{Satz}

\begin{Rechenregeln}{Koordinatentransformationen und Funktionaldet.}{}
	Wichtige Koordinatentransformationen und Funktionaldeterminanten\\
	
	Polarkoordinaten in $\R^2$
	\begin{alignat*}{4}
            x &= r \cos \varphi \quad &&0 \leq r < \infty \quad &&&dxdy = r \cdot drd\varphi\\
            y &= r \sin \varphi \quad &&0 \leq \varphi < 2\pi &&&
    \end{alignat*}
   	Elliptische Koordinaten $\R^2$
   	\begin{alignat*}{4}
            x &= r \cdot  a \cos \varphi\quad &&0 \leq r < \infty\quad &&&dxdy = a \cdot b \cdot  r \cdot drd\varphi\\
            y &= r \cdot b \sin \varphi\quad &&0 \leq \varphi < 2\pi &&&
   	\end{alignat*}
        Zylinderkoordinaten $\R^3$ \begin{alignat*}{4}
            x &= r \cdot  a \cos \varphi\quad &&0 \leq r < \infty &&&dxdydz = r \cdot drd\varphi dz\\
            y &= r \cdot b \sin \varphi\quad &&0 \leq \varphi < 2\pi\\
            z &= z\quad &&\-\infty \leq z < \infty\quad
    	\end{alignat*}
        Kugelkoordinaten $\R^3$ \begin{alignat*}{4}
            x &= r \cdot \sin \theta \cos \varphi \quad &&0 \leq r < \infty \quad &&&dxdydz = r^2 \sin \theta \cdot drd\theta d\varphi\\
            y &= r \cdot \sin \theta \sin \varphi \quad && 0 \leq \theta < \pi &&&\\
            z &= r \cos \theta \quad &&0 \leq \varphi < 2\pi &&&
        \end{alignat*}
\end{Rechenregeln}
