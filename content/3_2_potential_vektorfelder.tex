\subsection{Potential}

Intuition: Stärke der Änderung der Richtung der Vektoren in einem Vektorfeld.

\begin{Definition}{Potentialfelder und Potentiale}{}
	Ein Vektorfeld $\vec{v} : \Omega \subset \R^n \to \R^n$ heisst \textbf{Potentialfeld}, falls eine stetig differenzierbare Abbildung $\Phi : \Omega \subset \R^n \to \R$ existiert, sodass \[\vec{v} = \nabla \Phi\] gilt. Das skalare Feld $\Phi$ heisst dann \textbf{Potential} von $\vec{v}$. \textbf{Wichtig:} Es gibt sehr viele Vektorfelder, die sich nicht als Gradient eines skalaren Feldes schreiben lassen (also keine Potentialfelder sind)!
\end{Definition}

\begin{Rezept}{Potential finden}{}
	Sei $\vec{v} = \vek{f_1}{\vdots}{f_n}$ ein Vektorfeld. Dann muss für das Potential $\Phi$ stimmen:
	\[
		\vec v = \vek{\frac{\partial \Phi}{\partial x_1}}{\vdots}{\frac{\partial \Phi}{\partial x_n}}
	\]
\end{Rezept}

\subsection{Konservative Vektorfelder (= Potentialfelder)}

\begin{Satz}{Wegintegrale für Potentialfelder}{}
	Sei $\vec{v} : \Omega \subset \R^n \to \R^n$ ein Potentialfeld mit Potential $\Phi$. Dann gilt für jedes Wegintegrale entlang $\gamma$, dass
	\[
		\int_\gamma \vec{v} \cdot d\vec{s} = 
		\int_a^b \vec{v}(\gamma(t)) \cdot \gamma'(t) dt =
		\Phi(\gamma(b)) - \Phi(\gamma(a))
	\]
	Wir müssen also nur die Potentiale am Anfangs- und Endpunkt der Kurve auswerten! Damit sieht man auch gerade, dass für jede geschlossene Kurve das Wegintegrale eines Potentialfeldes gleich 0 ist. 
\end{Satz}

\begin{Diverses}{Zusammenfassung}{}
	Sei $\vec{v} : \Omega \subset \R^n \to \R^n$ ein stetig differenzierbares Vektorfeld und $\Omega$ einfach zusammenhängend. Folgende Aussagen sind äquivalent:
	\begin{itemize}
		\item $\vec{v}$ ist konservatives Vektorfeld
		\item $\vec{v}$ ist ein Potentialfeld
		\item Für alle geschlossene Kurven gilt $\oint \vec{v} \cdot d\vec{s} = 0$
		\item Das Integral $\int_\gamma \vec{v} \cdot d\vec{s}$ ist unabhängig vom Weg
	\end{itemize}
\end{Diverses}
