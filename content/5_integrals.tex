%% (c) 2019 Oliver Senn

\section{Integrale}
\subsection{Spezielle unbestimmte Integrale}
${\int (ax+b)^s\;dx= \frac{1}{a(s+1)}(ax+b)^{s+1} + C,\;s\neq -1}$\\
${\int \frac{1}{ax+b}dx=\frac{1}{a}\;log|ax+b|+C}$\\
${\int (ax^p+b)^sx^{p-1}\;dx = \frac{(ax^p+b)^{s+1}}{ap(s+1)}+C,\;s\neq -1, a \neq 0}$\\
${\int (ax^p+b)^{-1}x^{p-1}\;dx=\frac{1}{ap}log|ax^p+b|+C,\;a\neq 0,p\neq 0}$\\
${\int \frac{ax+b}{cx+d}dx=\frac{ax}{c}- \frac{ad-bc}{c^2} log|cx+d|+C}$\\
${\int \frac{1}{x^2+a^2}dx=\frac{1}{a}arctan(\frac{x}{a})+C}$\\
${\int \frac{1}{x^2-a^2}dx=\frac{1}{2a}log\Bigl|\frac{x-a}{x+a}\Bigl|}$\\
${\int \sqrt{a^2+x^2}\;dx=\frac{x}{2}\sqrt{a^2+x^2}+\frac{a^2}{2} log(x+\sqrt{a^2+x^2})+C}$\\
${\int \sqrt{a^2-x^2}\;dx= \frac{x}{2}\sqrt{a^2-x^2}+\frac{a^2}{2}arcsin\Bigl(\frac{x}{|a|}\Bigl)+C}$\\
${\int \sqrt{x^2-a^2}\;dx= \frac{x}{2}\sqrt{x^2-a^2}-\frac{a^2}{2}log|x+\sqrt{x^2-a^2}|+C}$\\
${\int \frac{1}{\sqrt{x^2-a^2}}\;dx=log(x+\sqrt{a^2+x^2})+C}$\\
${\int \frac{1}{\sqrt{x^2-a^2}}\;dx=log|x+\sqrt{x^2-a^2}|+C}$\\
${\int \frac{1}{\sqrt{a^2-x^2}}\;dx=arcsin\Bigl(\frac{x}{|a|}\Bigl)+C}$\\
${\int e^{kx}\;dx=\frac{1}{k}e^{kx}+C}$\\
${\int a^{kx}\;dx=\frac{1}{k*log(a)}a^{kx}+C}$\\
${\int e^{ax}p(x)\;dx=e^{ax}(a^{-1}p(x)-a^{-2}p'(x)+a^{-3}p''(x)-\dots}$\\
${+(-1)^na^{-n-1}p^{(n)}(x))+C}$, $a\neq 0$, p: Polynom n-ten Grades\\
${\int e^{kx}sin(ax+b)\;dx=\frac{e^{kx}}{a^2+k^2}\Bigl(k\;sin(ax+b)- a\;cos(ax+b)\Bigl)+C}$\\
${\int e^{kx}cos(ax+b)\;dx=\frac{e^{kx}}{a^2 + k^2}\Bigl(k\;cos(ax+b)+a\;sin(ax+b)\Bigl)+C}$\\
${\int log|x|\;dx= x(log|x|-1)+C}$\\
${\int x^k log(x)\;dx=\frac{x^{k+1}}{k+1}\Bigl(log(x)-\frac{1}{k+1}\Bigl)+C}$, $k\neq -1$\\
${\int x^{-1}log(x)\;dx= \frac{1}{2}(log(x))^2 + C}$\\
${\int sin(ax+b)\;dx=-\frac{1}{a}cos(ax+b)+C}$\\
${\int cos(ax+b)\;dx=\frac{1}{a}sin(ax+b)+C}$\\
${\int tan^2(x)\;dx= tan(x)-x+C}$\\
${\int \frac{1}{sin(x)}\;dx=log\Bigl|tan\frac{x}{2}\Bigl|+C}$\\
${\int \frac{1}{cos(x)}\;dx=log\Bigl|tan\Bigl(\frac{x}{2}+\frac{\pi}{4}\Bigl)\Bigl|+C}$\\
${\int \frac{1}{tan(x)}\;dx=log|sin(x)|+C}$\\
${\int coth\;dx=log|sinh(x)|+C}$\\
$\int \sin^2(ax)\;dx= \frac{x}{2}- \frac{\sin(2ax)}{4a} = \frac{x}{2} - \frac{\sin(ax) \cdot \cos(ax)}{2a}$\\
$\int \sin^3(ax)\;dx= - \frac{\cos(ax)}{a}- \frac{\cos^3(ax)}{3a}$\\
$\int x^2 \cdot \sin(ax)\;dx = \frac{2x \cdot \sin(ax)}{a^2} - \frac{a^2x^2-2 \cdot \cos(ax)}{a^3}$\\
$\int \frac{dx}{\sin(ax)}\;dx = - \frac{\cot(ax)}{a}$\\
$\int x \cdot \sin^2(ax)\;dx = \frac{x^2}{4} - \frac{x \cdot \sin(2ax)}{4a} - \frac{\cos(2ax)}{8a^2}$\\
$\int \frac{x \;dx}{\sin^2(ax)} = -\frac{x \cdot \cot(ax)}{a} + \frac{1}{a^2} \cdot \ln \abs*{\sin(ax)}$\\
$\int \cos^2(ax) \;dx = \frac{x}{2} + \frac{\sin(2ax)}{4a} = \frac{x}{2} + \frac{\sin(ax \cdot \cos(ax))}{2a}$\\
$\int \cos^3(ax) \;dx = \frac{\sin(ax)}{a} - \frac{\sin^3(ax)}{3a}$\\
$\int x \cdot \cos(ax) \;dx = \frac{\cos(ax)}{a^2} + \frac{x \cdot \sin(ax)}{a}$\\
$\int x^2 \cdot \cos(ax) \;dx = \frac{2x \cdot \cos(ax)}{a^2} + \frac{(a^2x^2 - 2) \cdot \sin(ax)}{a^3}$\\
$\int \frac{dx}{\cos(ax)} = \frac{1}{a} \cdot \ln \abs*{\tan(\frac{ax}{2} + \frac{\pi}{4})}$\\
$\int \frac{dx}{\cos^2(ax)} = \frac{\tan(ax)}{a}$\\
$\int x \cdot \cos^2(ax) \;dx = \frac{x^2}{4} + \frac{x \cdot \sin(2ax)}{4a} + \frac{\cos(2ax)}{8a^2}$\\
$\int \frac{x \;dx}{\cos^2(ax)} = \frac{x \cdot \tan(ax)}{a} + \frac{1}{a^2} \cdot \ln \abs*{\cos(ax)}$\\


\subsection{Spezielle bestimmte Integrale}
${\int_0^{2\pi} sin(mx)cos(nx)\;dx=0}$, $m,n\in \mathbb{Z}$\\
${\int_0^\infty \frac{sin(ax)}{x}\;dx=\frac{\pi}{2},\;a>0}$\\
${\int_0^\infty sin(x^2)\;dx=\int_0^\infty cos(x^2)\;dx=\frac{1}{2}\sqrt{\frac{\pi}{2}}}$\\
${\int_0^\infty e^{-ax}x^n\;dx= \frac{n!}{a^{n+1}},\;a>0}$\\
${\int_0^\infty e^{-ax^2}\;dx=\frac{1}{2}\sqrt{\frac{\pi}{a}},\;a>0}$\\

